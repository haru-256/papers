\documentclass[uplatex, dvipdfmx]{jsarticle}

    \usepackage{amsmath, amssymb, amsthm}
    \usepackage{bm}
    \usepackage{ascmac}
    \usepackage[hiresbb]{graphicx}
    \usepackage{algpseudocode, algorithm}
    
    \theoremstyle{definition}
    \newtheorem{theorem}{定理}
    \newtheorem*{theorem*}{定理}
    \newtheorem{definition}[theorem]{定義}
    \newtheorem*{definition*}{定義}
    % プリアンブル
    \title{Wassernstein GAN}
    \author{Martin Arjovsky et al.}
    \date{2017/1}
    \begin{document}
    \maketitle
    \abovedisplayskip=10.0pt% plus 4.0pt minus 6.0pt 
    \belowdisplayskip=10.0pt% plus 4.0pt minus 6.0pt % 
    % \setlength{\abovedisplayskip}{10pt} % 上部のマージン
    % \setlength{\belowdisplayskip}{5pt} % 下部のマージン      

    \section{Introduction}
    この論文が関係している問題は教師なし学習のそれである.主に問題は,確率分布を学習するということは
    どういうことなのか?である.古典的な回答は,確率密度を学ぶ事ということである.これはよく,パラメトリックな分布族$(P_{\theta})$を
    定義し,そしてデータにおいて尤度を最大にした$\theta$を見つけることによってなされる.


    
    \end{document}