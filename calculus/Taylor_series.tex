
\documentclass[uplatex, dvipdfmx]{jsarticle}

    \usepackage{amsmath, amssymb, amsthm}
    \usepackage{bm}
    \usepackage{ascmac}
    \usepackage[hiresbb]{graphicx}
    \usepackage{algpseudocode, algorithm}
    
    \theoremstyle{definition}
    \newtheorem{theorem}{定理}
    \newtheorem*{theorem*}{定理}
    \newtheorem{definition}[theorem]{定義}
    \newtheorem*{definition*}{定義}
    %%%%% argmax %%%%%
    \makeatletter
    \def\argmax{\mathop{\operator@font argmax}\limits}
    \makeatother

    %%%%% argmin %%%%%
    \makeatletter
    \def\argmin{\mathop{\operator@font argmin}\limits}
    
    % プリアンブル
    \title{Taylor series}
    \author{岡林}
    \date{2018/7/31}
    \begin{document}
    \maketitle
    \abovedisplayskip=10.0pt% plus 4.0pt minus 6.0pt 
    \belowdisplayskip=10.0pt% plus 4.0pt minus 6.0pt % 
    % \setlength{\abovedisplayskip}{10pt} % 上部のマージン
    % \setlength{\belowdisplayskip}{5pt} % 下部のマージン      

    \section{Abstract}
    テイラー展開は,式の形は覚えているものの,収束半径やら導出等をいちいち忘れてしまうので,
    メモ.

    \section{1変数のテイラー展開}
    \subsection{冪級数}
    今,ある任意の関数$f(x)$を無限次元多項式で表そうとする.つまり,$f(x)$は以下のように表すことができるとする.
    \begin{equation}
        f(x) = a_0 + a_1 x + a_2 x^{2} + \cdots + a_n x^{n} + \cdots
    \end{equation}
    \cite{masema} P98 より,ある任意の関数$f(x)$の接線は以下のように表すことができる.

    \begin{itembox}[l]{曲線の接線}
        微分可能な曲線$y = f(x)$上の任意の点$(a, f(a))$における接線の方程式は,次式で表される.
        \begin{equation}
            y = f'(a) (x - a) + f(a)
        \end{equation}
    \end{itembox}
    
    これは任意の点$a$の周りの$x$について,関数$f(x)$を1次多項式で近似していることを表している.
    次に2次多項式で近似してみることを考える.
    \begin{equation}
        f(x) \approx f(a) + f'(a) (x - a) + p (x-a)^{2} \qquad (x \fallingdotseq a の時) \label{eq:nizi}
    \end{equation}
    ここで,$p$ はある定数であるとする.この時,$p$がどんな値をとるか調べる.
    \ref{eq:nizi}の両辺を$x$で二回微分し,係数比較を行うと
    \begin{equation}
        p = \frac{f^{(2)}(a)}{2}
    \end{equation}
    となることがわかる.同様に3次多項式,4次多項式で表そうとする.これを続け,無限次元多項式で関数を近似すると
    \begin{equation}
        f(x) \approx \sum_{n=0}^{\infty} \frac{f^{n}}{n!} (x-a)^{n} \qquad (x \fallingdotseq a の時) 
    \end{equation}
    となる.これを$f(x)$の点$a$周りでのテイラー展開と呼ぶ.
    \subsection{テイラー展開}
    前節でテイラー展開による関数近似
    \begin{equation}
        f(x) \approx \sum_{n=0}^{\infty} \frac{f^{n}}{n!} (x-a)^{n} \qquad (x \fallingdotseq a の時)
    \end{equation}
    を見てきた.ここで,右辺の$n \leqq k$の部分和を考える.
    \begin{equation}
       \sum_{n=0}^{k} \frac{f^{n}}{n!} (x-a)^{n} \qquad (x \fallingdotseq a の時)
    \end{equation}
    上式と関数$f(x)$のテイラー展開との差を考える.この差は近似式の関数$f(x)$との誤差と考えることができる.




    
    

    % 参考文献
    \newpage
    \begin{thebibliography}{10}
        \bibitem{masema}
        馬場敬之 \\
        微分積分 キャンパス・ゼミ マセマ出版

        \bibitem{EMAN}
        EMANの物理学 (テイラー級数) : http://eman-physics.net/math/taylor.html

        \bibitem{Martin}
        Martin Arjovsky and L'eon Bottou. \\
        Towards principled methods for training generative adversarial networks. I \\
        https://arxiv.org/abs/1701.04862
    \end{thebibliography}
    
    \end{document}